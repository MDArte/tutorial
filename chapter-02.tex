\chapter{MDArte}
O MDArte é um \texttt{framework} voltado para o desenvolvimento de sistemas que
utiliza a abordagem dirigida a modelo (\texttt{MDD}) que implementa a
arquitetura baseada em modelo
(\texttt{MDA}\footnote{http://www.omg.org/mda/index.htm}) definida pela
\texttt{Object Management Group} (\texttt{OMG\footnote{http://www.omg.org/}}) em 2001.

No desenvolvimento orientado a modelos, os diagramas conceituais são utilizados
não somente para documentar e especificar o sistema a ser desenvolvido, mas
também como componentes para o desenvolvimento do mesmo. Esses modelos são
utilizados para a geração do código da aplicação e assim se tornam artefatos do
desenvolvimento.

O objetivo desse tipo de desenvolvimento é o aumento da produtividade
(reutilizando os modelos), simplificação do processo do design da arquitetura do
sistema que será desenvolvido devido às gerações realizadas utilizando modelos
como entrada e o aumento da comunicação com a equipe de desenvolvimento, de
análise e de domínio.

A arquitetura baseada em modelo (\texttt{MDA}) é uma iniciativa da \texttt{OMG}
que tem como intuito definir um padrão de arquitetura para o MDD. Assim, esse padrão passa a
ser seguido tanto pela comunidade quanto pela indústria de desenvolvimento.

O \texttt{AndroMDA} é um \texttt{framework} de código aberto que implementa as
transformações definidas pelo \texttt{MDA} e é dividido em núcleo e plugins
(cartuchos). O núcleo é responsável por realizar a leitura nos modelos e disponibilização de suas
informações para os cartuchos. Já os cartuchos definem os artefatos que de fato
deverão ser gerado para a aplicação a partir das informações modeladas. Cada
cartucho é organizado de forma a agregar características de mesma tecnologia.
O MDArte disponibiliza hoje alguns diferentes tipos de cartuchos: \texttt{EJB},
\texttt{Hibernate}, \texttt{Java}, \texttt{JUnit} e \texttt{Struts}.

O MDArte surgiu da necessidade de incorporar mais funcionalidades nas
transformações realizadas pelo o \texttt{framework}
\texttt{AndroMDA\footnote{http://www.andromda.org/}} e voltado para o
desenvolvimento de software para o governo brasileiro. Ele compreende de um conjunto de cartuchos
com diversas soluções de projetos e tendo a possibilidade de agregar novos
cartuchos dependendo das demandas do governo e da comunidade em geral.