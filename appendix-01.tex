\appendix
\chapter{Configuração do JBoss e acesso ao banco de dados}
\label{jboss-config}
Neste apêndice veremos como configurar as informações de acesso ao banco de
dados do nosso projeto, bem como demais configurações do nosso servidor de
aplicação (\texttt{JBoss}).

\section{Configuração das propriedades do projeto para acesso ao banco de dados}
Para se configurar o Banco de Dados é necessário modificar o arquivo
\texttt{project.properties} da raiz do projeto, onde se encontram as
propriedades que devem ser alteradas. Os arquivos
\texttt{project.properties} são arquivos onde são definidas propriedades que são
usadas pelo \texttt{MDArte} durante a sua exucação, este, na raiz do projeto,
especificamemte concentra propriedades de acesso ao banco e de \texttt{deploy}
do projeto.

Abaixo estão as propriedades do arquivo de
configuração para cada um dos Bancos de Dados:

\begin{itemize}
	\item [Oracle] \hfill
	\begin{itemize}
		\item dataSource.driver.jar=\textdollar{}\{env.JBOSS\_HOME\}/server/default/lib/ojdbc14.jar
		\item dataSource.driver.class=oracle.jdbc.driver.OracleDriver
		\item sql.mappings=Oracle9i
		\item hibernate.db.dialect=org.hibernate.dialect.Oracle9Dialect
	\end{itemize}
	\item [SQLServer] \hfill
	\begin{itemize}
		\item dataSource.driver.jar=\textdollar{}\{env.JBOSS\_HOME\}/server/default/lib/jtds-1.1.jar
		\item dataSource.driver.class=net.sourceforge.jtds.jdbc.Driver
		\item sql.mappings=MSSQL
		\item hibernate.db.dialect=org.hibernate.dialect.SQLServerDialect
	\end{itemize} 
	\item [Postgres] \hfill
	\begin{itemize}
		\item dataSource.driver.jar=\textdollar{}\{env.JBOSS\_HOME\}/server/default/lib/postgresql.jar
		\item dataSource.driver.class=org.postgresql.Driver
		\item defaultHibernateGeneratorClass=sequence
		\item sql.mappings=PostgreSQL
		\item hibernate.db.dialect=org.hibernate.dialect.PostgreSQLDialect
	\end{itemize}
	\item [MySQL] \hfill
	\begin{itemize}
		\item dataSource.driver.jar=\textdollar{}\{env.JBOSS\_HOME\}/server/default/lib/mysql-connector-java-5.1.6-bin.jar
		\item dataSource.driver.class=com.mysql.jdbc.Driver
		\item defaultHibernateGeneratorClass=native
		\item sql.mappings=MySQL
		\item hibernate.db.dialect=org.hibernate.dialect.MySQLDialect
	\end{itemize}
\end{itemize}

Tais propriedades são as responsáveis por definir qual banco de dados estará
sendo usado no projeto, bem como qual biblioteca será usada para a comunicação
com o banco. Propriedades como a \texttt{url} do banco, nome de usuário, senha
etc. não precisam ser alteradas uma vez que, a seguir, as definiremos diretamente na
configuração do \texttt{JBoss}.

\section{Configuração do JBoss}
Agora veremos como configurar o servidor \texttt{JBoss} e qual a finalidade dos
arquivos utilizados para tal fim. 

\subsection{Configuração dos datasources utilizados pelo JBoss}
Para a configuração dos datasources utilizados pelo \texttt{JBoss} é preciso
criar ou alterar o arquivo responsável por registrar e gerenciar tais fontes de dados.
O arquivo que deve estar localizado no diretório
\texttt{JBOSS\_HOME/server/default/deploy/}, com formação do nome terminando com
\texttt{-ds.xml} (ex.: \texttt{aplicacoes-ds.xml)}, que deve ter a \texttt{tag}
\texttt{<local-tx-datasource>} preenchida de acordo com as informações 
fornecidas no arquivo \texttt{<projeto>/project.properties}.

Exemplo (usando banco \texttt{Postgres}):

\begin{framed}
	\lstinputlisting[language=xml]{files/sistemaacademico-ds.xml}
\end{framed}

Repare que no exemplo anterior, o nome do \texttt{Data Source} é
\texttt{sistemaacademicoDS}, que deve ser o mesmo nome informado no arquivo
\texttt{project.properties} no diretório raiz do projeto. Aqui também definimos algumas outras propriedades do
datasource \texttt{sistemaacademicoDS} que haviam ficado em aberto antes como a
\texttt{url} de conexão com o servidor de banco de dados, usuario e senha.

\subsection{Configuração do acesso das aplicações ao banco de dados}
Para tal, alteraremos o arquivo \texttt{login-config.xml}, localizado no
diretório \texttt{JBOSS\_HOME/server/default/conf/}. Alteraremos o arquivo
adicionando uma tag \texttt{<application-police name='<nomeAplicacao>'>} com
seus campos devidamente preenchidos como no exemplo abaixo, onde temos a
configuração para o \texttt{Sistema Acadêmico} deste tutorial.

\begin{framed}
	\lstinputlisting[language=xml]{files/sistemaacademico-login-config.xml}
\end{framed}

O arquivo alterado concentra as informações de login para as diversas aplicações
sendo rodadas no servidor, informações como: datasource que contém os dados de
usuário usados no \texttt{login}, \texttt{query} a ser rodada para buscar os
dados de usuário, algoritmo de \texttt{hash} da senha etc. As informações
presentes nesse arquivo permitirão a aplicação do \texttt{Sistema Acadêmico} se
conectar a base de dados e validar o usuário no momento de \texttt{login}.

\chapter{Configurando repositório externo do Maven}
\label{maven-config}