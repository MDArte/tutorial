\chapter{Tipos de dados Especiais}
Neste capítulo veremos alguns tipos de dados especiais que o MDArte implementa e
como utilizá-los, a fim de facilitar a comunicação com o banco de dados e o
tratamento dos dados da aplicação.

\section{Hstore}
O hstore é um tipo de dados disponível no Postgres que permite, para uma mesma
coluna de uma linha na tabela, relacionar valores de texto à chaves, que também
são textos, tal qual um Map em Java.
Para mais informações sobre o funcionamento do hstore, consulte
http://www.postgresql.org/docs/9.4/static/hstore.html.
Nesta seção veremos como utilizar o tipo hstore nas entitades de um sistema
desenvolvido com o MDArte.

\subsection{Criando um atributo com o tipo hstore}
No diagrama de classes da sua camada de domínio, abra a especificação do
atributo que deverá ser do tipo hstore. No campo type, selecione o tipo Hstore,
como na imagem abaixo:
\begin{figure}[H]
	\centering
	\includegraphics[width=290pt,height=220pt]{files/imgs/hstore-0000.png}
	\caption{Configuração do parâmetro matrícula da classe Estudante.}
	\label{config_parametro}
\end{figure}

O resultado na classe que representa a entidade, será o seguinte:
\begin{figure}[H]
	\centering
	\includegraphics[width=260pt,height=180pt]{files/imgs/hstore-0001.png}
	\caption{Entidade com o atributo nome já com o tipo hstore.}
	\label{config_parametro}
\end{figure}

Feito isto, vá na pasta raiz do projeto e regere o mesmo usando o comando
\texttt{maven}. O \texttt{MDArte} irá então regerar as classes da camada de
domínio, bem como os \texttt{scripts SQL} corrspondentes aos novos atributos da
entidade. Você provavelmente precisará rodar novamente o \texttt{script
schema-create.sql}, ou alterar manualmente a tabela, a fim de que o atributo
escolhido como \texttt{hstore} tem a tipagem correta na tabela do banco. Você
provavelmente precisará também reinserir os dados da sua aplicação, caso a
tabela já tenha alguma informação.

\subsection{Acessando um atributo hstore}
Quando um atributo de uma entidade é modelado como sendo do tipo
\texttt{Hstore}, o \texttt{MDArte} o mapeará para o código da aplicação como um
\texttt{Map}, mais especificamente um \texttt{HashMap}. Como resultado disso, o
atributo será gerado nas classes de \texttt{<nomeDaEntidade>TO} e
\texttt{<nomeDaEntidade>Abstract} com o tipo \texttt{Map}.

Para acessar e alterar os valores do \texttt{Map} são gerados os seguintes
métodos:

\texttt{public void set<nomeDoAtributo>(String key, String value)} - Adiciona um
novo valor no indice indicado.

\texttt{public void set<nomeDoAtributo>(Map<String,String> map)} - Substitui o
\texttt{map} da entidade por um \texttt{clone} do \texttt{map} passado por
parâmetro.

\texttt{public String get<nomeDoAtributo>(String key)} - Retorna o valor do
\texttt{map} correspondente à chave passada por parâmetro.

\texttt{public Map<String,String> get<nomeDoAtributo>()} - Retorna um
\texttt{clone} do \texttt{map} existente na classe da entidade.

\texttt{public void remove<nomeDoAtributo>(String key)} - Remove do \texttt{map}
a chave correspondente \texttt{string} passada por parâmetro.

\subsection{Hstore e persistência de dados}
A persistência dos dados com o \texttt{Hstore} é bem parecida com a dos demais
atributos, todas as alterações feitas nos indíces do \texttt{map} serão
persistidas mediante o uso dos métodos \texttt{update} ou \texttt{insert} na 
classe de serviço da entidade. No entanto, a filtragem de dados baseada no
atributo \texttt{Hstore} não está disponível, ou seja, sempre que executada uma
filtragem de dados em uma entidade, caso haja um atributo do tipo
\texttt{Hstore}, os valores do mesmo serão desconsiderados na operação.