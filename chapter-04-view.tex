\section{Reformulação da View}
Nesta seção, trataremos da nova estrutura e organização da camada de visão do
MDArte.

A base da estrutura da \texttt{view} do MDArte encontra-se nos \texttt{layouts},
que são arquivos que definem a estrutura das páginas que definem, através da
definição de \texttt{tags} \texttt{html} e da importação demais arquivos
contendo partes menores da \texttt{view}.

O \texttt{layout} \texttt{default} do \texttt{MDArte} é definido nos arquivos de
nome \texttt{main-layout2.jsp}. Além disso, há também os arquivos
\texttt{main-layout-open.jsp}, responsável por definir o \texttt{layout} das
páginas que não estão submetidas à controle de acesso, e
\texttt{simple-layout.jsp}, que configura o \texttt{layout} de páginas simples
ou até mesmo de porções de hypertexto retornadas via \texttt{AJAX}.

Este conjunto de arquivos de \texttt{layout} é gerado \texttt{para cada módulo
Web} do sistema, a partir do \texttt{default} do sistema, dando liberdade ao
desenvolvedor de, se necessário, adaptar a aparência de cada módulo
independentemente.

\subsection{Estrutura básica dos arquivos de layout}
O \texttt{layout} define algumas divisões e componentes da página, cada
\texttt{view} em específico importará o seu respectivo \texttt{layout} com essas
definições, implementando-as através da importação dos arquivos referentes a
cada seção da página.

Algumas das principais seções definidas no \texttt{layout} das páginas são as
seguintes:
\begin{itemize}
  \item 'title' - Define o título da página, através de uma propriedade definida
  no arquivo \texttt{application-resources.csv};
  \item 'style' - Seção onde são importados as folhas de estilo da página, sendo
  estas \texttt{[nome-da-pagina].css} (css gerado automaticamente) e
  \texttt{[nome-da-pagina]-impl.css} (css implementado manualmente);
  \item 'vars' - Seção onde é importado o arquivo contendo as variáveis da
  página, de nome \texttt{[nome-da-pagina]-vars.jspf};
  \item 'tables' - Seção onde é importado o arquivo que contém as tabelas usadas
  nas páginas, de nome \texttt{[nome-da-pagina]-tabelas.jspf}. Neste arquivo de
  tabelas são importadas as diversas tabelas que podem existir numa página,
  sendo estas geradas em arquivos de nome
  \texttt{[nome-da-pagina]-[nome-da-tabela].jspf};
  \item 'other' - Seção onde são importados arquivos adicionais utilizados na
  \texttt{view}, em especial os arquivos de \texttt{JavaScript} gerado
  (\texttt{[nome-da-pagina].js}) e implementado manualmente
  (\texttt{[nome-da-pagina]-impl.js});
  \item 'body' - Seção onde é inserido o conteúdo principal da página, através
  da importação dos arquivos \texttt{[nome-da-pagina]-container.jsp}, arquivo
  que contém a estrutura do corpo da página e importa outros arquivos que
  contém unidades menores da \texttt{view}.
\end{itemize}

\subsection{O arquivo de container e sua estrutura}
Arquivo onde é inserido o conteúdo principal da página, através da importação
dos arquivos \texttt{[nome-da-pagina]-container.jsp}, arquivo que contém a estrutura
do corpo da página e importa outros arquivos que contém unidades menores da
\texttt{view}.

Este arquivo contém um formulário usado para o envio das informações das ações
da \texttt{view} para o servidor. 

Esse formulário é declarado no
\texttt{container}, porém dentro desse elemento são importados outros dois
arquivos, \texttt{[nome-da-view]-actions.jsp} e
\texttt{[nome-da-view]-fields.jsp}, que agrupam, respectivamente os botões e
campos do formulário.
