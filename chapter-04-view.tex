\section{Reformulação da View}
Nesta seção, trataremos da nova estrutura e organização da camada de visão do
MDArte.

A base da estrutura da \texttt{view} do MDArte encontra-se nos \texttt{layouts},
que são arquivos que definem a estrutura das páginas que definem, através da
definição de \texttt{tags} \texttt{html} e da importação demais arquivos
contendo partes menores da \texttt{view}.

O \texttt{layout} \texttt{default} do \texttt{MDArte} é definido nos arquivos de
nome \texttt{main-layout2.jsp}. Além disso, há também os arquivos
\texttt{main-layout-open.jsp}, responsável por definir o \texttt{layout} das
páginas que não estão submetidas à controle de acesso, e
\texttt{simple-layout.jsp}, que configura o \texttt{layout} de páginas simples
ou até mesmo de porções de hypertexto retornadas via \texttt{AJAX}.

Este conjunto de arquivos de \texttt{layout} é gerado \texttt{para cada módulo
Web} do sistema, a partir do \texttt{default} do sistema, dando liberdade ao
desenvolvedor de, se necessário, adaptar a aparência de cada módulo
independentemente.
