\appendix
\chapter{Criando um novo cartucho}

O \texttt{MDArte} é composto por uma série de componentes, dentre estes, temos
os cartuchos, um dos mais essenciais e motivo de estudo neste capítulo. Os
cartuchos são responsáveis por prover a habilidade de processar um modelo
utilizando Templates, a fim de gerar o código-fonte. Veremos neste capítulo que
os cartuchos são compostos por vários artefatos, tal como: arquivos XML de
configuração; arquivos para os Templates Velocity; classes Java para as
Metafacades; metamodelos de dados armazenados em um arquivo XMI; etc. E
principalmente, como estes artefatos interagem entre si.

O processo de criação de um cartucho para o MDArte é bastante simples.
Basicamente sendo necessário:
\begin{itemize}
\item Criar uma estrutura de diretório padronizada, a fim de armazenar os
artefatos devidamente.
\item Criar o descritor de configuração do cartucho (andromda-cartridge.xml).
\item Criar os Templates Velocity.
\item Compilar o \texttt{MDArte} junto com .
\item Instalar o cartucho desenvolvido em um repositório para o Maven.
\end{itemize}

Em alguns casos, veremos que o uso de Metafacades personalizadas será
necessário. A criação de Metafacades será, então, abordada mais à frente,
fazendo com que o processo de criação possuirá etapas adicionais.
