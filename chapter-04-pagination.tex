\section{Estratégia de paginação}

Estratégias de paginação foram criadas para comportar diferentes tipos de tabela
e paginação. Nas versões anteriores, a paginação era feita do mesmo jeito para
as tabelas Struts e Ajax porém elas tinham estruturas diferentes o que gerava
erros nas tabelas. Dessa forma, não poderíamos usar a tabela Ajax pois tinhamos
como prioridade ter a tabela Struts funcionando. Com as estratégias de paginação
podemos utilizar qualquer tipo de tabela sem ter que alterar a estrutura dos
DAOs e damos a opção para o desenvolvedor customizar a paginação do jeito que
quiser.

Nesta seção demonstraremos como utilizar estratégias de paginação. Primeiro,
apresentaremos a classe abstrata \texttt{PaginationStrategy}:

\lstinputlisting[language=java, frame=single, breaklines=true] {files/java/PaginationStrategy.java}

A classe abstrata \texttt{PaginationStrategy} é gerado no pacote util do
projeto. Nele você seta a página a ser obtida,  o número de linhas por
página e o número de páginas a serem retornadas pela \texttt{query} e
\texttt{criteria}.

Nela há a função \texttt{paginateResult} que possui duas assinaturas diferentes.
O desenvolvedor tem que implementar essas funções caso deseje criar a sua
própria estratégia. Um exemplo de implementação de estratégia é o
\texttt{PaginationSimple.java} que é utilizado para a paginação da tabela ajax.

\lstinputlisting[language=java, frame=single, breaklines=true] {files/java/PaginationSimple.java}

Além dessa estratégia, criamos outras duas: \texttt{PaginationDisplaytag} para as tabelas Struts e \texttt{NoPagination} caso não seja preciso de paginação.

A pasta em que sua implementação deve ser criada é
<projeto>/common/src/java/<pacote\_do\_projeto>/util e o import a ser feito é
\texttt{<pacote\_do\_projeto>.util.<nome>}.

Demonstramos um simples exemplo para instanciar uma estratégia abaixo:

\begin{lstlisting}[language=java, frame=single, breaklines=true]
	import br.mdarte.exemplo.academico.util.PaginationDisplaytag;
	import br.mdarte.exemplo.academico.util.Constantes;
	
	Integer pagina = ((Integer)request.getAttribute(Constantes.PARAMETRO_PAGINA)); //Struts 1
	Collection exemplos = ServiceLocator.instance().getExemploHandlerBI().recuperaExemplos(new PaginationDisplaytag(paginacao));
\end{lstlisting}