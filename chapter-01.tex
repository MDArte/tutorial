\chapter{Preparação do Ambiente}

Nesta seção detalharemos o processo de preparação do ambiente de desenvolvimento com o AndroMDA, onde serão enumeradas as ferramentas utilizadas e seus respectivos procedimentos de instalação. 
Ferramentas necessárias:
\begin{itemize}
  \item Máquina Virtual Java - JDK
  \item JBoss (versão 4.2.3-GA)
  \item Maven (versão 1.0.2) 
  \item Magic Draw (versão 9.5)
  \item Eclipse Indigo (versão 3.7.1)
\end{itemize}

\section{JDK}

É necessário que o JDK esteja instalado no computador. O download pode ser feito em http://java.sun.com/ ou utilizando algum repositório, como mostra abaixo:

\begin{lstlisting}[language=bash]
sudo add-apt-repository ppa:webupd8team/java
sudo apt-get update
sudo apt-get install oracle-java6-installer
\end{lstlisting}

\section{JBoss}

É necessário que o JBoss esteja instalado no computador. O download pode ser feito em h\hypertarget{ttp://www.jboss.com/}{ttp://www.jboss.com/}. Após a instalação do JBoss, é necessário configurar a variável \texttt{JBOSS\_HOME}, a qual deve especificar o caminho de instalação do JBoss. Esse caminho para o JBoss é necessário para a instalação de aplicações (deployment).

Neste tutorial será considerado que o jboss estará instalado na pasta abaixo.

\begin{lstlisting}[language=bash]
/home/<user>/Work/programs/jboss
\end{lstlisting}

\section{Maven}
O Maven\footnote{\hypertarget{http://maven.apache.org/}{http://maven.apache.org/}} é uma ferramenta de automação e gerenciamento de projetos. O download pode ser feito através do endereço \hypertarget{http://maven.apache.org/start/download.html}{http://maven.apache.org/start/download.html} e sua instalação consiste em descompactar o arquivo obtido em um diretório local. 

A versão compatível é a 1.02. O Maven, durante sua execução, faz acesso a repositórios remotos, de onde poderão ser obtidos diversos artefatos necessários às tarefas de automação. Por exemplo bibliotecas (arquivos \texttt{*.jar}) necessárias para compilação e execução de um projeto podem ser automaticamente obtidas. 

Para especificar o repositório que o Maven deve acessar é necessário criar um arquivo, chamado \texttt{build.properties}, no diretório home do usuário, por exemplo \texttt{/home/<usuario>}. 

Abaixo temos exemplos do arquivo build.properties para uso com repositórios externos, em repositório externo que possui um proxy para acesso à internet, e com um repositório proxy configurado na rede local. 

1. Repositórios remoto quando não é necessário utilizar proxy para acesso à internet: 

\begin{lstlisting}[language=bash]
maven.repo.remote=http://www.ibiblio.org/maven,http://team.andromda.org/maven 
\end{lstlisting}

2. Repositórios remoto quando é necessário utilizar proxy para acesso à internet. No exemplo, o IP do proxy é \texttt{10.0.2.15}: 

\begin{lstlisting}[language=bash]
maven.repo.remote=http://www.ibiblio.org/maven,http://team.andromda.org/maven maven.proxy.host=10.0.2.15 
maven.proxy.port=8080 
\end{lstlisting}

3. Repositório proxy na rede local. O repositório proxy contém as bibliotecas \texttt{*.jar} que o Maven poderia requisitar no repositório remoto na Internet. Esse repositório deverá ser utilizado quando não é possível ou não é desejável o acesso ao repositório remoto:

\begin{verbatim}
maven.repo.remote=http://<host>:<port> 
\end{verbatim}

Onde \texttt{<host>} é o nome da máquina utilizada como proxy e \texttt{<port>} é o número da porta do serviço do proxy. Por exemplo: \texttt{maven.repo.remote=http://146.164.34.92/repositorio/}

É importante observar que o Maven, de acordo com as tarefas executadas, irá fazer o download dos artefatos necessários e guardá-los em um cache local na estação de trabalho em um diretório chamado \texttt{.maven} localizado no diretório home de cada usuário da estação. 

Para evitar o acesso a servidores na Internet é possível a instalação de um proxy específico do Maven na rede local. Neste tutorial de configuração, não abordaremos a configuração desse proxy.

Neste tutorial será considerado que o jboss estará instalado na pasta abaixo.

\begin{verbatim}
/home/<user>/Work/programs/maven
\end{verbatim}

\section{Varíaveis de Ambiente}

Adicionar no final do arquivo \texttt{/home/<usuario>/.bashrc} o seguinte código: 

\begin{lstlisting}[language=bash]
if [ -f ~/.bashrc_mdarte ]; then
 . ~/.bashrc_mdarte
fi
\end{lstlisting}

Criar o script onde ficará todas as variáveis do ambiente. 

\begin{lstlisting}[language=bash]
touch ~/.bashrc_mdarte
\end{lstlisting}
	
Editar o arquivo \texttt{/home/<usuario>/.bashrc\_mdarte} adicionando os seguintes valores. 

\begin{lstlisting}[language=bash]
#MDArte Configurations

export MAVEN_OPTS=-Xmx1024m

export JAVA_HOME=/usr/lib/jvm/java-6-oracle/

if [ -d ~/Work/programs/maven ] ; then
  export MAVEN_HOME=~/Work/programs/maven
fi

if [ -d ~/Work/programs/maven ] ; then
 export PATH=\$PATH:~/Work/programs/maven/bin
fi

if [ -d ~/Work/programs/jboss ] ; then
  export JBOSS_HOME=~/Work/programs/jboss
fi
\end{lstlisting}

Após será necessário reiniciar a sessão do usuário para essas variáveis estarem no sistema ou utilizar o seguinte comando abaixo.

\begin{lstlisting}[language=bash]
source ~/.bashrc
\end{lstlisting}

\section{MDArte}

O MDArte, na verdade, não é um aplicativo, mas sim um conjunto de bibliotecas de classes. Em nosso processo de desenvolvimento, utilizaremos o MDArte como um plugin do Maven. O Maven, por sua vez, possui um mecanismo próprio para obtenção de plugins. Através de parâmetros na linha de comando podemos especificar ao Maven qual plugin queremos instalar e ele se encarrega de buscar este plugin no(s) repositório(s) para o(s) qual(is) estiver configurado. 

No caso do plugin do MDArte, o seguinte comando deve ser executado para a instalação (ao copiar o comando, verificar se foi copiado corretamente, inclusive os hifens): 

\begin{lstlisting}[language=bash]
maven plugin:download -DgroupId=andromda -DartifactId=maven-andromdapp-plugin-coppetec -Dversion=3.1.1.3.4.19-RC2
\end{lstlisting}
	
Após a execução desse comando o Maven terá instalado o plugin do AndroMDA no cache local do usuário e tarefas referentes ao MDArte poderão ser executadas através do Maven. 

Eventualmente, dependendo das tarefas executadas, o Maven poderá buscar outros artefatos nos repositórios, contudo isso será feito de forma transparente e automática.

\section{MagicDraw}

O download do MagicDraw pode ser feito em http://www.magicdraw.com. 

O MagicDraw é uma ferramenta para modelagem em UML e é recomendada para uso com o MDArte devido a seu suporte a diagramas de atividade, utilizados pelo cartucho BPM4Struts. Ainda, para que os modelos sejam corretamente utilizados pelo MDArte eles deverão conter estereótipos específicos, disponíveis através de um profile fornecido com o MDArte, que será mostrado com mais detalhes na seção “Iniciando o projeto no MagicDraw”.

\section{Eclipse}

O download do Eclipse pode ser feito em http://www.eclipse.org/. 

Durante a geração de um projeto, o MDArte gerará automaticamente os arquivos de configuração \texttt{.project} e \texttt{.classpath} de um projeto Eclipse. Esses arquivos podem ser usados diretamente para importação do projeto ao Eclipse. O \texttt{.classpath} é o arquivo onde será indicado as bibliotecas para o eclipse que serão utilizados pelo projeto. Assim, o eclipse saberá completar as informações automaticamente. Já o \texttt{.project} é uma descrição das opções do projeto.

Citation of Einstein paper~\cite{Einstein}.