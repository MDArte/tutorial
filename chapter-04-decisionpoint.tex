\section{Pontos de decisão}

Pontos de decisão são criados quando sua aplicação precisa tomar decisões
dependendo do resultado de uma ação prévia. Utilizamos uma forma de modelagem
ofericida pela especificação UML.

Nela você desenha uma linha de transição a partir da ação conectando-a ao ponto
de decisão. Um ponto de decisão é desenhado como um losango em UML. Já que uma
decisão tem pelo menos dois resultados diferentes, o ponto de decisão terá
múltiplas transições para diferentes ações.

Usaremos como exemplo o seguinte diagrama de atividade

Exemplo de modelagem:


Exemplo de código:

\begin{lstlisting}[language=java, frame=single, breaklines=true]
	public String decisionPoint(DecisionPointForm form, ViewContainer container) throws Exception
	{
		if (form.getIdResolucao() != null) {
			return "COM_RESOLUCAO";
		}
		else {
			return "SEM_RESOLUCAO";
		}
	}
\end{lstlisting}