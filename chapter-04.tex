\chapter{Funcionalidades do MDArte}

Neste capítulo iremos explorar algumas funcionalidades que já existem no MDArte,
a fim de agilizar e simplificar o processo de desenvolvimento. Para tal
alteraremos os modelos de \texttt{CRUD} gerados automaticamente pelo
\texttt{MDArte}. Para evitar que as alterações feitas sejam sobrescritas por
engano, vá no diagrama de classe que descreve as entidades do Banco de Dados,
abra a especificação da classe \texttt{Estudante}, selecione a aba
\texttt{stereotypes} e remova o estereótipo \texttt{«Manageable»}.

\section{Campo com Autocomplete}

Nesta seção veremos como implementar um \texttt{autocomplete} para um
determinado campo de texto. Iremos transformar o campo \texttt{matricula} do
caso de uso \texttt{Consulta Estudante} em um campo com \texttt{autocomplete}.

O modelo inicial do caso de uso \texttt{Consulta Estudante} no  \texttt{CRUD}
para a entidade estudante pode ser visto na imagem \ref{modelo_consulta_estudante}:

\begin{figure}[H]
	\centering
	\includegraphics[width=350pt,height=300pt]{files/imgs/tutorial-mdarte-0028.png}
	\caption{Modelo do caso de uso Consulta Estudante.}
	\label{modelo_consulta_estudante}
\end{figure}

Abriremos a especificação da \texttt{transition} que sai da \texttt{front end
view} de nome \texttt{preencha os campos}, clicaremos no botão \texttt{edit}, no
\texttt{fieldset} \texttt{trigger}. Na aba \texttt{parameters}, da janela
\texttt{signal event specification}, que será aberta automaticamente, dê duplo
clique no nome do parâmetro \texttt{matricula} e será então aberta a
especificação do parâmetro. Selecione então a aba \texttt{tagged values},
selecione o \texttt{tagged value}
\texttt{@andromda.presentation.web.view.field.type} e clique no botão
\texttt{create value}. Selecione então a opção \texttt{autocomplete}, como na
imagem \ref{field_type_autocomplete}.

\begin{figure}[H]
	\centering
	\includegraphics[width=350pt,height=300pt]{files/imgs/tutorial-mdarte-0027.png}
	\caption{Adicionando o field type 'autocomplete' à um campo da view.}
	\label{field_type_autocomplete}
\end{figure}

Agora executaremos o seguinte comando no terminal na raiz do projeto:
\begin{lstlisting}[language=bash]
maven mda -Dprojeto=sistemaacademico-geral-Estudante
\end{lstlisting}

Feito isto, o \texttt{MDArte} gerará automaticamente toda a estrutura
responsável por receber e tratar as requisições assíncronas para o preenchimento do
\texttt{autocomplete}, restando ao desenvolver apenas implementar no
\texttt{ControleImpl} a filtragem dos valores retornados, de acordo com o valor
do campo. Para isto, criaremos, na classe
\texttt{ConsultaEstudanteControleImpl}, um método seguindo o seguinte padrão 
\texttt{protected String[]
<nome-do-campo><nome-do-caso-de-uso>AutoComplete(java.lang.String query,
org.andromda.bpm4struts.ViewContainer container)}. Vejamos abaixo um exemplo de
implementação para o \texttt{autocomplete} do nosso campo de \texttt{matrícula}:

\lstinputlisting[language=java, frame=single, breaklines=true] {files/java/autocomplete.java}

Agora executaremos o seguinte comando para compilar e dar \texttt{deploy} no
\texttt{Sistema Acadêmico} :

\begin{lstlisting}[language=bash, frame=single, breaklines=true]
maven compile deploy
\end{lstlisting}

Feito isto, daremos \texttt{start} no \texttt{JBoss} e abriremos o sistema e
faremos login. Na tela \texttt{Preencha os Campos} do caso de uso
\texttt{Consulta Estudante} podemos agora verificar o \texttt{autocomplete}
funcionando, como na imagem \ref{exemplo_autocomplete}.

\begin{figure}[H]
	\centering
	\includegraphics[width=350pt,height=300pt]{files/imgs/tutorial-mdarte-0029.png}
	\caption{Exemplo de autocomplete.}
	\label{exemplo_autocomplete}
\end{figure}
\section{Estratégia de paginação}

Estratégias de paginação foram criadas facilitar a filtragem de coleções de objetos e podendo dar uma maior flexibilidade e facilidade ao desenvolvimento de tabelas de dados. Com as estratégias de paginação podemos utilizar qualquer tipo de tabela sem ter que alterar a estrutura dos DAOs para definir a paginação e como consequência facilita ao desenvolvedor adequar a solução para cada caso.

Nesta seção demonstraremos como utilizar estratégias de paginação. Primeiro, apresentaremos a classe abstrata \texttt{PaginationStrategy}:

\lstinputlisting[language=java, frame=single, breaklines=true] {files/java/PaginationStrategy.java}

A classe abstrata \texttt{PaginationStrategy} é gerado no pacote util do projeto. Nele você seta a página a ser obtida,  o número de linhas por página e o número de páginas a serem retornadas pela \texttt{query} e
\texttt{criteria}.

Nela existe uma função \texttt{paginateResult} que possui duas assinaturas diferentes. O desenvolvedor tem que implementar essas funções caso deseje criar a sua própria estratégia. Esse método é chamado pelo DAO no momento da criação da \texttt{query}. Um exemplo de implementação de estratégia é o \texttt{PaginationSimple.java} que é utilizado para a paginação da tabela ajax.

\lstinputlisting[language=java, frame=single, breaklines=true] {files/java/PaginationSimple.java}

Além dessa estratégia, criamos outras duas: \texttt{PaginationDisplaytag} para as tabelas Struts e \texttt{NoPagination} caso não seja preciso de paginação.

Existe a possibilidade de criar uma própria paginação. Essa classe deverá ser ser criada em <projeto>/common/src/java/<pacote\_do\_projeto>/util e o import a ser feito é \texttt{<pacote\_do\_projeto>.util.<nome>}. Essa classe precisará extender a classe \texttt{PaginationStrategy} e implementar os métodos abstratos.

Demonstramos um simples exemplo para instanciar uma estratégia abaixo:

\begin{lstlisting}[language=java, frame=single, breaklines=true]
	import br.mdarte.exemplo.academico.util.PaginationDisplaytag;
	import br.mdarte.exemplo.academico.util.Constantes;
	
	Integer pagina = ((Integer)request.getAttribute(Constantes.PARAMETRO_PAGINA)); //Struts 1
	Collection exemplos = ServiceLocator.instance().getExemploHandlerBI().recuperaExemplos(new PaginationDisplaytag(paginacao));
\end{lstlisting}
\section{Internacionalização}

Internacionalizar aplicações web é cada vez mais uma tarefa corriqueira de todo
desenvolvedor web e é um dos processos importantes para o aumento da
acessibilidade do sistema. O framework deve permitir mecanismo para facilitar a
utilização de diversas línguas. Para tal, foi construido um conjunto de
ferramentas para facilitar esse processo. A maioria dos frameworks web tem a sua
maneira particular de prover esse mecanismo, mas o que muita gente desconhece é
que existe uma forma padrão de fazer isso,  definida na especificação do Java EE, através da JSP Standard TagLibs.

O MDArte utiliza esse mecanismo e já provê ela configurada. Assim, a
especificação de novas linguas se torna uma tarefa ainda mais fácil.

Segue abaixo os passos para a definições de novas linguas.

\subsection{Mensagens}

Cada arquivo properties conterá todas as traduções do sistema. Todos os textos
do sistemas serão representados por uma \texttt{key}. Cada \texttt{key} e sua
respectiva mensagem, ou seja, \texttt{label} serão listadas em cada arquivo como
properties. Abaixo segue um exemplo:

\begin{lstlisting}[language=xml, frame=single, breaklines=true]
	label.key=Mensagem
\end{lstlisting}

Todas os recursos modelados no diagrama de atividades serão gerados com uma \texttt{key}.
Caso essa \texttt{key} não esteja no arquivo properties, o sistema utilizará a
\texttt{key} como \texttt{label} contudo sem o ponto.

Cada uma das possibilidades será abordada.

\subsubsection{Título de uma Página}

Todo título de página é gerado com uma key para o desenvolvedor possa definir no
custom-resources. 

\begin{lstlisting}[language=html, frame=single, breaklines=true]
	<tiles:put name="title" type="string">
    	<bean:message key="pagina.exemplo.title"/>
	</tiles:put>
\end{lstlisting}

\subsubsection{Campo ou Botão de uma Página}

Todo campo ou botão é gerado com uma key definida pelo o nome do caso de uso
somado com o nome do campo/botão. Segue abaixo um exemplo.

\begin{lstlisting}[language=html, frame=single, breaklines=true]
	<td class="field">
	<s:set name="__value" value="#session.form.nome"/>
	<div id="divnomeConsultaCursoUC" class="textfield field">
    	<label class="textfieldLabel" for="nome"><bean:message key="consulta.curso.uc.preencha.campos.consulta.curso.param.nome"/></label>
    	<s:textfield id="nomeConsultaCursoUC" name="nome" label="%{getText('consulta.curso.uc.preencha.campos.consulta.curso.param.nome')}" value="%{#session.form.nome}" title="" styleId="consultaCursoNome" />
</div>
\end{lstlisting}

\subsubsection{Exception}

Toda mensagem carregando uma \texttt{exception} também é substituída quando a
\texttt{key} na \texttt{exception} é encontrada no \texttt{custom-resource}.
Segue abaixo um exemplo.

\begin{lstlisting}[language=java, frame=single, breaklines=true]
	throw new Exception("ocorre.erro.esperado");

	throw new Exception("ocorre.erro.nao.esperado", exception);
\end{lstlisting}

Existem algumas funções que não interrompem a execução, mas possui o mesmo
efeito do \texttt{exception}. Segue abaixo um exemplo.

\begin{lstlisting}[language=java, frame=single, breaklines=true]
	saveErrorMessage(request, "informando.erro.key"); //Struts 1
	saveWarningMessage(request, "informando.aviso.key"); //Struts 1
	saveSuccessMessage(request, "informando.sucesso.key"); //Struts 1
	saveErrorMessage("informando.erro.key", container); //Struts 2
	saveWarningMessage("informando.aviso.key", container); //Struts 2
	saveSuccessMessage("informando.sucesso.key", container); //Struts 2
\end{lstlisting}

A função \texttt{saveErrorMessage} gera a seguinte mensagem na página:

\begin{figure}[H]
	\centering
	\includegraphics[scale=0.75]{files/imgs/internacionalizacao-00.png}
	\caption{Mensagem de erro}
	\label{mensagem_erro}
\end{figure}

\subsubsection{Passagem de Parâmetros}

Existe a possibilidade de passar parâmetros para a mensagem. Será passando um
\texttt{array} de \texttt{string} e na mensagem existirá marcadores informando
onde será colocado o conteúdo do \texttt{array}.

\textbf{Exemplo:}

\texttt{custom-resources.properties}

\begin{lstlisting}[language=xml, frame=single, breaklines=true]
	label.key=Mensagem com parametro {0}
\end{lstlisting}

\texttt{Código:}

\begin{lstlisting}[language=java, frame=single, breaklines=true]
	String[] parametro = new String[1];
	parametro[0] = "param1";
	saveErrorMessage("label.key", parametro, container); //Struts 2
	//saveErrorMessage(request, "label.key", parametro); //Struts 1
\end{lstlisting}

Assim será exibido a mensagem \texttt{Mensagem com
parametro param1} na figura abaixo:

\begin{figure}[H]
	\centering
	\includegraphics[scale=0.75]{files/imgs/internacionalizacao-01.png}
	\caption{Mensagem de erro com parâmetros}
	\label{mensagem_erro_parametro}
\end{figure}

\subsection{Arquivo de Configurações}

Edite o arquivo \textbf{<DiretorioProjeto>}/mda/conf/andromda.xml e na
propriedade \texttt{languages} informe os locales que serão utilizados.

Exemplo:

\begin{lstlisting}[language=xml, frame=single, breaklines=true]
	<property name="languages">pt,en,fr</property>
\end{lstlisting}

Após a geração utilizando essa configuração será criado novos três arquivos: 

\begin{lstlisting}[language=xml, frame=single, breaklines=true]
	custom-resources_en.properties
	custom-resources_pt.properties
	custom-resources_fr.properties
\end{lstlisting}

Além do arquivo já criado no geração da aplicação:

\begin{lstlisting}[language=xml, frame=single, breaklines=true]
	custom-resources.properties
\end{lstlisting}

Automaticamente o sistema irá detectar qual locale do browser que o usuário está
utilizando e utilizará o locale correto. Caso não exista o locale pré-definido,
o sistema utilizará o \texttt{custom-resources} padrão.

O desenvolvedor poderá forçar um locale específico pelo código. Exemplo: 

\begin{lstlisting}[language=java, frame=single, breaklines=true]
	//Recuperando o Locale
	Locale locale = (Locale) request.getSession().getAttribute("org.apache.struts.action.LOCALE");

	//Definindo um Locale
	request.getSession().setAttribute("org.apache.struts.action.LOCALE", new Locale("pt", "BR"));
\end{lstlisting}

\section{Modo de Operação}

Usar modo operação permite que o desenvolvedor mostre campos, botões e outros
artefatos de maneira condicionais no frontend. A idéia básica por trás do modo
de operação é a seguinte:

Se um usuário chegar na página através de um botão A, você exibe tal página sem
certas opções. Se um usuário chegar na mesma tela através de um botão B, você
exibe a mesma página com as opções omitidas no caso anterior. O modo de
operação permite que isso seja feito.

A seguir demonstramos um exemplo:

\subsection{Exemplo}

Dados a entidade USUARIO, onde usuario tem os atributos ( nome, email, sexo ['M'
| 'F'], certificadoDeReservista). Porém para sexo igual a 'F' certificado é null
e para 'M' certificado deve ser diferente de null.

Suponhamos agora que o desenvolver deseje mostrar esse usuário. Porém quando o
usuário for masculino deveremos mostrar o certificado e quando for feminino não
mostraremos. Para isso utilizaremos o modo de operação. 

Primeramente defineremos no campo certificado de reservista o modo de operação
CERTIFICADO. Uma vez definido o campo com o modo de operação, ele aparecerá
somente se for invocado desse modo operação, no nosso caso, CERTIFICADO.

Podemos invocar o modo de operação a partir do fluxo ou do controlador para que
apareça o campo certificado. 

Primeiro, demonstramos como ele será invocado no controlador.

\subsection{Invocando pelo Controlador} 

Para definir o valor que tem que ser invocado para mostrar o campo com modo
operação, fazemos:
 
\begin{enumerate}

\item Abrimos a transição, vamos para trigger e clicamos em edite, vamos na aba
parameters e selecionamos o campo a ser colocado o modo de operação, caso o
campo exista clique em edit, caso não exista o campo clique em add.

\begin{figure}[H]
	\centering
	\includegraphics[scale=0.75]{files/imgs/operation-mode-00.png}
	\caption{Criando o campo}
	\label{criando_campo_modo_operacao}
\end{figure}

\item Agora clique na aba Tagged Values:
@andromda.presentation.view.operation.mode.
\item Em seguida clique em Create Value.
\item Agora digite o nova do valor que voce deseja, que no nosso caso será
CERTIFICADO

\begin{figure}[H]
	\centering
	\includegraphics[scale=0.75]{files/imgs/operation-mode-01.png}
	\caption{Adicionando o modo de operação do campo}
	\label{aicionando_modo_operacao_campo}
\end{figure}

\item Pelo Eclipse vá ao ponto de implementação da classe de controle que tem
que disparar o modo de operação e adicione o seguinte código:

\begin{lstlisting}[language=java, frame=single, breaklines=true]
if( estudante.getSexo().getValue().equals("M") ){
	adicionaModoOperacao("CERTIFICADO",container);// <---- Aqui disparamos o MODO OPERACAO
	form.setCertificado(estudante.getCertificado());
}
\end{lstlisting}

\end{enumerate}

Quando chamarmos essa classe de controle, nós iremos invocar o modo de operação
da view caso a condição seja satisfeita nesse caso.

Em seguida, mostraremos como fazer a invocação do modo de operação pelo fluxo.

\subsection{Invocando pelo Fluxo}

Fazendo por fluxo, temos que colocar um valor no valor etiquetado
@andromda.presentation.action.input.operation.mode de uma transição que está
indo para o caso de uso (indo para um estado final) que contém o campo com o
mesmo modo de operação.

\begin{figure}[H]
	\centering
	\includegraphics[scale=0.75]{files/imgs/operation-mode-02.png}
	\caption{Adicionando o modo de operação do campo}
	\label{aicionando_modo_operacao_campo}
\end{figure}

Repita o passo 1 da invocação por controlador para ter um campo em que o
funcionamento do modo de operação possa ser testado. Apenas isso precisa ser
feito para fazer modo de operação por fluxo.

Existe um valor no valor etiquetado
@andromda.presentation.action.keep.operation.mode que é também usado em uma
transição que está indo para um estado. Se colocar o valor dele como true você
manterá para o próximo caso de uso o modo de operação recebido de outro caso de
uso.

\subsection{Implementação}

\subsubsection{Sistema}

\subsubsection{Componentes}

\subsection{Na JSP\ldots}

Nas PaginasPersonalizadas, o modo de operação aparece como a tag

<security:containsOperationMode value=“NomeDoModoDeOperacao”>

que envolve o componente definido pelo caso de uso.


\subsection{Regra de ouro do modo de operação}

    Se o componente não possuir nenhum modo de operação definido, ele será
    exibido sempre,  independente se existe ou não um modo de operação definido no sistema.
    
    Se o componente tiver um modo de operação definido,  o componente só será
    exibido na página se o modo de operação do sistema for igual
    
    Você pode definir mais de um modo de operação,  tanto para o sistema quanto
    para um componente. Basta separar os nomes dos modos de operação por vírgula.


\section{Pontos de decisão}

Pontos de decisão são criados quando sua aplicação precisa tomar decisões
dependendo do resultado de uma ação prévia. Utilizamos uma forma de modelagem
ofericida pela especificação UML.

Nela você desenha uma linha de transição a partir da ação conectando-a ao ponto
de decisão. Um ponto de decisão é desenhado como um losango em UML. Já que uma
decisão tem pelo menos dois resultados diferentes, o ponto de decisão terá
múltiplas transições para diferentes ações.

Usaremos como exemplo o seguinte diagrama de atividade

Exemplo de modelagem:


Exemplo de código:

\begin{lstlisting}[language=java, frame=single, breaklines=true]
	public String decisionPoint(DecisionPointForm form, ViewContainer container) throws Exception
	{
		if (form.getIdResolucao() != null) {
			return "COM_RESOLUCAO";
		}
		else {
			return "SEM_RESOLUCAO";
		}
	}
\end{lstlisting}
\section{Tabela assíncrona (JTable)}

Nesta seção veremos como implementar uma tabela \texttt{assíncrona}
(\texttt{JTable}) usando o \texttt{MDArte}. Para tal, vamos considerar como
ponto de partida o modelo do caso de uso \texttt{Consulta Estudante} conforme as
alterações feitas no tópico anterior. Certifique-se de ter removido o
estereótipo \texttt{«Manageable»} da classe \texttt{Estudante} no diagrama de
classes que descreve a \texttt{Camada de domínio}, a fim de evitar que o
\texttt{CRUD} seja re-gerado, sobrescrevendo assim as alterações que faremos.

Veremos agora, por subseções, algumas das funcionalidades disponíveis na tabela
assíncrona.

\subsection{Implementando uma tabela simples}

Para implementar uma tabela assíncrona, precisamos primeiramente abrir o modelo
do caso de uso \texttt{Consulta Estudante}, abriremos então a especificação da
\texttt{transition} que sai da \texttt{action} \texttt{ConsultandoEstudante}
para \texttt{Front End View} \texttt{'ResultadoConsulta'}, clicar no botão
\texttt{edit} no \texttt{fieldset} \texttt{trigger}, iremos então na aba
\texttt{parameters}, na janela \texttt{signal event specification}, aberta
automaticamente. Daremos então um duplo clique no nome do parâmetro
(\texttt{estudantes}), que representa a tabela que será mostrada na view.
Selecionaremos a aba \texttt{tagged values} selecione o \texttt{tagged value}
\texttt{@andromda.presentation.web.view.field.table.type} e clique no botão
\texttt{create value}. Selecione então a opção \texttt{jtable}, como na
imagem \ref{table_type_jtable}.

\begin{figure}[H]
	\centering
	\includegraphics[width=350pt,height=300pt]{files/imgs/tutorial-mdarte-0030.png}
	\caption{Mudando tipo da tabela para 'jtable'.}
	\label{table_type_jtable}
\end{figure}

Agora executaremos o seguinte comando no terminal na raiz do projeto:

\begin{lstlisting}[language=bash, frame=single, breaklines=true]
maven mda -Dprojeto=sistemaacademico-geral-Estudante
\end{lstlisting}

Feito isto, o MDArte já terá gerado toda a estrutura necessária pela recepção e
tratamento das requisições assíncronas da tabela, restando ao desenvolvedor
implementar somente dois métodos: um para indicar o numero total de elementos a
serem exibidos na tabela, usado para fazer a paginação da mesma, e outro para
retornar a coleção de objetos a serem exibidos na página atual da tabela.

A assinatura do método para o carregamento da tabela segue o seguinte padrão:

\begin{lstlisting}[language=java, frame=single, breaklines=true]
protected Collection load[nome-da-view][nome-da-tabela]Table(PaginationStrategy
paginacao, String propriedade, Boolean desc, ViewContainer container)
\end{lstlisting}

A assinatura do método que retorna o total de elementos na tabela segue o
seguinte padrão:

\begin{lstlisting}[language=java, frame=single, breaklines=true]
protected Integer get[nome-da-view][nome-da-tabela]TableLength(PaginationStrategy
paginacao, String propriedade, Boolean desc, ViewContainer
container) throws Exception
\end{lstlisting}

Também será necessário alterar o método \texttt{carregaDados} do
\texttt{ControllerImpl} para disponibilizar as informações preenchidas na
\texttt{PreenchaCampos} para serem usadas pela tabela disponível na
\texttt{view} \texttt{ResultadoConsulta}.

Abaixo o exemplo de implementação para a tabela \texttt{estudantes} da
\texttt{view ResultadoConsulta} no caso de uso \texttt{Consulta Estudante}.

\lstinputlisting[language=java, frame=single, breaklines=true] {files/java/JTableSimples.java}

Não se esqueça de fazer os \texttt{imports} necessários de acordo com as
alterações feitas. Se o seu \texttt{.classpath} estiver corretamente
configurado, você pode usar o atalho \texttt{Ctrl + Shift + O}, do eclipse, que
importará automaticamente todas as classes que estão faltando.

Agora, compilaremos e daremos deploy do código da aplicação, com o seguinte
comando:

\begin{lstlisting}[language=bash, frame=single, breaklines=true]
maven compile deploy
\end{lstlisting}

Abriremos agora a view \texttt{PreenchaCampos} e digitaremos um valor para
filtragem dos dados da entidade \texttt{Estudante}. Como na imagem
\ref{preencha_campos_tabela_async_simples}.

\begin{figure}[H]
	\centering
	\includegraphics[width=260pt,height=180pt]{files/imgs/tutorial-mdarte-0031.png}
	\caption{Filtrando dados da tabela assíncrona de Estudantes.}
	\label{preencha_campos_tabela_async_simples}
\end{figure}

Na imagem \ref{resultado_consulta_tabela_async_simples}, podemos ver a tabela
carregada com os dados resultantes da filtragens.

\begin{figure}[H]
	\centering
	\includegraphics[width=460pt,height=200pt]{files/imgs/tutorial-mdarte-0032.png}
	\caption{Filtragem dos dados da tabela assíncrona de Estudantes.}
	\label{resultado_consulta_tabela_async_simples}
\end{figure}

Note que o funcionamento da tabela se dá por requisições assíncronas
ao servidor, acabando com a necessidade de recarregar a página para alterar seus
dados. 

O \texttt{MDArte} já dá suporte a diversos tipos de chamadas assíncronas que se
possa querer fazer fazer com a tabela, no entanto, caso haja a necessidade de
mais informações sobre o funcionamento da mesma, a documentação pode ser
acessada \href{http://www.jtable.org/Home/Documents}{aqui}.

\subsection{Implementando filtragem assíncrona da tabela}
Nesta seção veremos como criar uma action ajax que reflita nos dados exibidos
por uma tabela ajax. A título de exemplo, faremos uma tela com um formulário e
um botão que, uma vez clicado, recarregará a tabela filtrando-a de acordo com os
dados do formulário. Tomaremos como base a tabela criada no exemplo de criação
de tabelas.

O modelo portanto começará assim:

\begin{figure}[H]
	\centering
	\includegraphics[width=180pt,height=260pt]{files/imgs/tutorial-mdarte-0040.png}
	\caption{Modelagem da filtragem assíncrona da tabela.}
	\label{modelando_filtragem_assincrona}
\end{figure}

Modelaremos então uma \texttt{transition} saindo da \texttt{FrontEndView} que
contém a nossa tabela e retornando para a mesma \texttt{view}, que será
interpretada pelo \texttt{MDArte} como uma ação assíncrona na \texttt{view}.
Abriremos então sua especificação e iremos na aba \texttt{tagged values} e
selecionaremos o \texttt{tagged value}
\texttt{@andromda.presentation.web.action.async.table} e daremos a ele o valor
\texttt{[nome-da-tabela]} (\texttt{estudantes}, nesse caso), esse \texttt{tagged
value} indica qual tabela será afetada pela ação assíncrona, nos permitindo ter
mais de uma de tabela assíncrona na mesma tela, tendo \texttt{actions} que
afetem somente uma tabela sem afetar a outra. Iremos então na aba
\texttt{general} e clicaremos no botão \texttt{edit} no \texttt{fieldset}
\texttt{'trigger'}, daremos o nome que desejarmos o \texttt{trigger} criado, no
exemplo fo idado o nome de \texttt{“filtrar”}, e selecionaremos seu tipo como
\texttt{signal}. Iremos então na aba \texttt{parameters}, ainda na especificação
do \texttt{trigger} e criaremos os parâmetros necessários para o processamento
da requisição, neste exemplo colocaremos só os parâmetros \texttt{matrícula} e
\texttt{nome}, a título de ilustração, e definiremos seus tipos como
\texttt{String}.

O modelo ficará assim:
\begin{figure}[H]
	\centering
	\includegraphics[width=340pt,height=300pt]{files/imgs/tutorial-mdarte-0036.png}
	\caption{Modelagem da filtragem assíncrona da tabela.}
	\label{modelando_filtragem_assincrona}
\end{figure}

Executaremos agora o seguinte comando para validar o modelo e regerar o
sistema:
\begin{lstlisting}[language=bash, frame=single, breaklines=true]
maven mda -Dprojeto=sistemaacademico-geral-Estudante
\end{lstlisting}

Agora adaptaremos, no \texttt{ControleImpl}, os métodos responsáveis pelo
carregamento da tabela, com a assinatura \texttt{public final Collection
load[nome-da-view][nome-da-tabela]Table}, e pelo retorno do número máximo de
elementos na mesma, com a assinatura \texttt{public final Integer
get[nome-da-view][nome-da-tabela]TableLength}, a fim de implementarmos o filtro.
Cada parâmetro da \texttt{trigger} pertencente à \texttt{transition} modelada
será adicionado, em ordem, à lista de parâmetros de cada método, logo antes do
parâmetro \texttt{ViewContainer container}. De acordo com o nosso exemplo, o
código ficará assim:
\lstinputlisting[language=java, frame=single, breaklines=true]{files/java/JTableFiltro.java}

Executaremos agora o seguinte comando para compilar e dar \texttt{deploy} no
sistema:
\begin{lstlisting}[language=bash, frame=single, breaklines=true]
maven compile deploy
\end{lstlisting}

Restartando o servidor e abrindo o caso de uso \texttt{Consulta Estudante},
veremos o seguinte formulário:
\begin{figure}[H]
	\centering
	\includegraphics[width=340pt,height=300pt]{files/imgs/tutorial-mdarte-0042.png}
	\caption{Modelagem da filtragem assíncrona da tabela.}
	\label{modelando_filtragem_assincrona}
\end{figure}

\section{Criação de Componente customizado}

Nesta seção veremos como criar um campo customizado em uma tela. Iremos partir
do modelo de \texttt{CRUD} gerado automaticamente pelo \texttt{MDArte}, para a
entidade \texttt{Estudante}, e faremos as alterações necessárias para a criação
de um novo componente. O componente a ser desenvolvido, somente a título de
exemplo, será um campo texto com máscara para \texttt{CPF}.

Podemos ver o estado inicial do modelo na imagem \ref{modelo_consulta_estudante_custom}.

\begin{figure}[H]
	\centering
	\includegraphics[width=350pt,height=300pt]{files/imgs/tutorial-mdarte-0028.png}
	\caption{Modelo do caso de uso Consulta Estudante.}
	\label{modelo_consulta_estudante_custom}
\end{figure}

Abriremos então a especificação da \texttt{transition}
\texttt{'consultaEstudante'}, clicaremos no botão \texttt{'edit'}, no
\texttt{fieldset} \texttt{'trigger'}, selecionaremos então a aba
\texttt{'parameters'}, da janela aberta quando clicamos o botão anterior, e
clicaremos então no botão \texttt{add}.

Preencheremos então os dados do campo conforme a imagem
\ref{dados_campo_custom_cpf}, SEM, no entanto, clicar no botão \texttt{Ok}.

\begin{figure}[H]
	\centering
	\includegraphics[width=350pt,height=300pt]{files/imgs/tutorial-mdarte-0033.png}
	\caption{Dados do campo 'cpf'.}
	\label{dados_campo_custom_cpf}
\end{figure}

Ainda na mesma janela, selecionaremos a aba \texttt{tagged values},
selecionaremos o \texttt{tagged value} 
\texttt{@andromda.presentation.web.view.field.type}, clicaremos no botão
\texttt{create value} e selecionaremos a opção \texttt{'custom'}, no campo
\texttt{combobox} que será exibido, como na imagem \ref{parametro_cpf_custom}.

\begin{figure}[H]
	\centering
	\includegraphics[width=350pt,height=300pt]{files/imgs/tutorial-mdarte-0034.png}
	\caption{Selecionando tipo custom para o campo 'cpf'.}
	\label{parametro_cpf_custom}
\end{figure}

Salvaremos então o modelo e digitaremos os seguintes comandos para regerar o
modelo:

\begin{lstlisting}[language=bash, frame=single, breaklines=true]
maven mda -Dprojeto=sistemaaacademico-geral-Estudante
\end{lstlisting}

Feito isto, o \texttt{MDArte} gerará um arquivo no padrão
\texttt{<nome-do-campo>.jsp}, neste caso, \texttt{cpf.jsp}, no caminho
\texttt{<nome-sistema>/web/<modulo-web>/src/jsp/<caminho-do-pacote-base>/web/<modulo-web>/<nome-caso-de-uso>},
mas você também pode encontrá-lo, no eclipse, através do comando
\texttt{ctrl+shift+r}, digitando o nome do arquivo na janela que é aberta por
esse comando.

Aberto o arquivo, adicionaremos a este o seguinte código \texttt{jsp}:

\lstinputlisting[language=html, frame=single, breaklines=true]{files/jsp/cpf.jsp}

O \texttt{html} adicionado será importado para a tela do sistema no espaço do
formulário dedicado ao campo \texttt{cpf}.

Adicionado o \texttt{jsp} do nosso componente, uma vez que se trata de um campo
de texto com um determinado comportamento (formatar a entrada no modelo do CPF),
precisamos agora adicionar um mecanismo de controle para o comportamento do
campo. Para tal, utilizaremos o \texttt{framework} para \texttt{javascript}
JQuery, uma vez que este já vem com o \texttt{MDArte}, além do fato de o
\texttt{JQuery} já possuir uma funcionalidade nativa que faça isso.

Para adicionar código \texttt{Javascript} manualmente a uma \texttt{view}
precisamos abrir o arquivo \texttt{<nome-da-view>-impl.js}, no mesmo caminho
do arquivo \texttt{jsp} alterado acima. O arquivo alterado é destinado ao código
adicionado pelo desenvolvedor para customizar o comportamento da aplicação, não
sendo sobrescrito durante a geração. Certifique-se, portanto, de estar
adicionando o seu código nestes pontos de implementação, a fim de não perdê-lo
na próxima geração.

Adicionaremos agora o seguinte código \texttt{JavaScript} ao arquivo
\texttt{preencha-campos-impl.js}:

\lstinputlisting[language=c, frame=single, breaklines=true]{files/js/preencha-campos-impl.js}

Executaremos agora o seguinte comando para compilar e dar \texttt{deploy} no
projeto:
\begin{lstlisting}[language=bash, frame=single, breaklines=true]
maven compile deploy
\end{lstlisting}

Abrindo o sistema e acessando a \texttt{view} \texttt{'Preencha Campos'} do caso
de uso \texttt{Consulta Estudante}, podemos conferir o resultado do nosso
componente \texttt{custom} de nome \texttt{cpf}, como na imagem
\ref{parametro_cpf_custom_exemplo}.
\begin{figure}[H]
	\centering
	\includegraphics[width=280pt,height=200pt]{files/imgs/tutorial-mdarte-0035.png}
	\caption{Resultado de componente custom de nome 'cpf'.}
	\label{parametro_cpf_custom_exemplo}
\end{figure}
\section{Reformulação da View}
Nesta seção, trataremos da nova estrutura e organização da camada de visão do
MDArte.

A base da estrutura da \texttt{view} do MDArte encontra-se nos \texttt{layouts},
que são arquivos que definem a estrutura das páginas que definem, através da
definição de \texttt{tags} \texttt{html} e da importação demais arquivos
contendo partes menores da \texttt{view}.

O \texttt{layout} \texttt{default} do \texttt{MDArte} é definido nos arquivos de
nome \texttt{main-layout2.jsp}. Além disso, há também os arquivos
\texttt{main-layout-open.jsp}, responsável por definir o \texttt{layout} das
páginas que não estão submetidas à controle de acesso, e
\texttt{simple-layout.jsp}, que configura o \texttt{layout} de páginas simples
ou até mesmo de porções de hypertexto retornadas via \texttt{AJAX}.

Este conjunto de arquivos de \texttt{layout} é gerado \texttt{para cada módulo
Web} do sistema, a partir do \texttt{default} do sistema, dando liberdade ao
desenvolvedor de, se necessário, adaptar a aparência de cada módulo
independentemente.

\subsection{Estrutura básica dos arquivos de layout}
O \texttt{layout} define algumas divisões e componentes da página, cada
\texttt{view} em específico importará o seu respectivo \texttt{layout} com essas
definições, implementando-as através da importação dos arquivos referentes a
cada seção da página.

Algumas das principais seções definidas no \texttt{layout} das páginas são as
seguintes:
\begin{itemize}
  \item 'title' - Define o título da página, através de uma propriedade definida
  no arquivo \texttt{application-resources.csv};
  \item 'style' - Seção onde são importados as folhas de estilo da página, sendo
  estas \texttt{[nome-da-pagina].css} (css gerado automaticamente) e
  \texttt{[nome-da-pagina]-impl.css} (css implementado manualmente);
  \item 'vars' - Seção onde é importado o arquivo contendo as variáveis da
  página, de nome \texttt{[nome-da-pagina]-vars.jspf};
  \item 'tables' - Seção onde é importado o arquivo que contém as tabelas usadas
  nas páginas, de nome \texttt{[nome-da-pagina]-tabelas.jspf}. Neste arquivo de
  tabelas são importadas as diversas tabelas que podem existir numa página,
  sendo estas geradas em arquivos de nome
  \texttt{[nome-da-pagina]-[nome-da-tabela].jspf};
  \item 'other' - Seção onde são importados arquivos adicionais utilizados na
  \texttt{view}, em especial os arquivos de \texttt{JavaScript} gerado
  (\texttt{[nome-da-pagina].js}) e implementado manualmente
  (\texttt{[nome-da-pagina]-impl.js});
  \item 'body' - Seção onde é inserido o conteúdo principal da página, através
  da importação dos arquivos \texttt{[nome-da-pagina]-container.jsp}, arquivo
  que contém a estrutura do corpo da página e importa outros arquivos que
  contém unidades menores da \texttt{view}.
\end{itemize}

\subsection{O arquivo de container e sua estrutura}
Arquivo onde é inserido o conteúdo principal da página, através da importação
dos arquivos \texttt{[nome-da-pagina]-container.jsp}, arquivo que contém a estrutura
do corpo da página e importa outros arquivos que contém unidades menores da
\texttt{view}.

Este arquivo contém um formulário usado para o envio das informações das ações
da \texttt{view} para o servidor. 

Esse formulário é declarado no
\texttt{container}, porém dentro desse elemento são importados outros dois
arquivos, \texttt{[nome-da-view]-actions.jsp} e
\texttt{[nome-da-view]-fields.jsp}, que agrupam, respectivamente os botões e
campos do formulário.

\section{Estereótipo OpenAccess}

Ao criar aplicações utilizando o MDArte, é oferecido ao usuário a opção de
utilizar a camada de segurança já implementada no MDArte. Ao escolher essa
opção, a aplicação MDArte só poderá ser acessada por usuários cadastrados.
Contudo, exitem aplicações em que o acesso público a algumas páginas da é
um requisito.

Para tornar a página pública você deve atribuir o estereótipo
\texttt{OpenAccess} ao caso de uso referente a página ou ao pacote que contém o caso de uso. Atribuir o
estereótipo a pacotes acima dos que tem os estereótipos \texttt{ModuloWeb} ou
\texttt{ModuloWebPai} não tornará a página pública pois o sistema finaliza a
checagem até pacotes com estes estereótipos.

O esetereótipo \texttt{OpenAccess} também é aplicado em pacotes da camada de
serviço. Isso é devido a casos de páginas públicas precisarem utilizar serviços
de acesso a banco de dados sem precisarem se logar. O estereótipo
\texttt{OpenAccess} deve ser aplicado em pacotes com estereótipo
\texttt{ModuloServico}

\begin{figure}[H]
	\centering
	\includegraphics[width=350pt,height=300pt]{files/imgs/openaccess-00.png}
	\caption{Exemplo de uso do estereótipo OpenAccess.}
	\label{open_access}
\end{figure}

No exemplo acima, todos os casos de uso do pacote \texttt{estudante} são
públicos. Se aplicasse o estereótipo no pacote \texttt{web} nada aconteceria. Se fosse apenas no pacote
\texttt{consultaPais}, apenas esse caso de uso seria público. Além disso, os
serviços dos pacotes \texttt{estudante} e \texttt{nota} são públicos.

